\documentclass[a4paper, titlepage, openany, oneside, 12pt]{article}
\usepackage{pfc}
\usepackage{url}
\usepackage{rotating}

\begin{document}
%%%%%%%%%%%%%%%%%%%%%%%%%%%%%%%%%%%%%%%%%%%%%%%%%%%%%%%%%%%%%%%%%%%%%%
%%%                       DADOS PESSOAIS                           %%%
%%%%%%%%%%%%%%%%%%%%%%%%%%%%%%%%%%%%%%%%%%%%%%%%%%%%%%%%%%%%%%%%%%%%%%

\titulo{T�tulo da Proposta ainda Indefinido} 	% coloque aqui o t�tulo do trabaho
\area{Intelig�ncia Artificial}
\subarea{Algoritmos Evolutivos}
\autor{Salviano Ludg�rio Felipe Gomes}
\orientador{Dr. Alexsandro Santos Soares}
\ano{2010}%modifique o ano
\mes{Mar�o}%modifique o m�s

%%%%%%%%%%%%%%%%%%%%%%%%%%%%%%%%%%%%%%%%%%%%%%%%%%%%%%%%%%%%%%%%%%%%%%
%%%                     ELABORA DOCUMENTO                          %%%
%%%%%%%%%%%%%%%%%%%%%%%%%%%%%%%%%%%%%%%%%%%%%%%%%%%%%%%%%%%%%%%%%%%%%%

\pagestyle{empty}
%\fazcapa % imprime a capa. Basta comentar a linha para que a capa n�o seja impressa.
\newpage
\vfill

\newpage
\vfill

\pagestyle{plain}

\pagenumbering{roman}
\newpage
\pagenumbering{arabic}

%%%%%%%%%%%%%%%%%%%%%%%%%%%%%%%%%%
%%% INCLUA AQUI SUAS SE��ES %%%
%%%%%%%%%%%%%%%%%%%%%%%%%%%%%%%%%%

\newpage %nova pagina
\pagenumbering{arabic}%numera��o em algarismo arabicos..

\section{}

\section{Aptid�o}

O trabalho proposto ser� desenvolvido sob aptid�o cient�fica, cuja finalidade � estudar o problema que envolve a automa��o
na gera��o de hor�rios escolares sob a vis�o de um procedimento de otimiza��o baseado no comportamento de abelhas.

%> Este projeto se enquadra na aptid�o cient�fica,
%> A aptid�o a ser explorada neste PFC � de natureza cient�fica

\section{Descri��o do Problema}

CCCCCCCCCCCCCCCCCCCCCCCC

\cite{Perone2006}

\section{Fundamenta��o Te�rica}

\cite{FALEIROS2007}

\section{Objetivos}
XXXXXXXXXXXXXXXXXXXXX
\section{Resultados Esperados}
FFFFFFFFFFFFFFFFFFFFFF
\section{Cronograma}
\begin{center}
\footnotesize
\begin{table}
\begin{tabular}{|p{8cm}|c|c|c|c|c|c|c|c|c|c|c|} \hline
\multicolumn{0}{|c|}{Atividade} & \multicolumn{10}{|c|}{2010} \\ \cline{2-11}
%\hline
&\begin{sideways}Mar\end{sideways}
&\begin{sideways}Abr\end{sideways}
&\begin{sideways}Mai\end{sideways}
&\begin{sideways}Jun\end{sideways}
&\begin{sideways}Jul\end{sideways}
&\begin{sideways}Ago\end{sideways} 
&\begin{sideways}Set\end{sideways}
&\begin{sideways}Out\end{sideways} 
&\begin{sideways}Nov\end{sideways}
&\begin{sideways}Dez\end{sideways} \\
\hline                  %% 2  3  4  5  6  7  8  9  10 11 
Levantamento Bibliogr�fico &x &x &x &x &x &x &x &x &x &x     \\\hline
Estudos das Bibliografias  &  &x &x &x &x &x &x &x &x &x     \\\hline
Estudo sobre ICALL         &  &x &x &x &x &x &x &x &x &x     \\\hline
Estudo das t�cnicas de 
Intelig�ncia Artificial    &  &  &x &x &x &x &x &x &x &x     \\\hline
Estudo sobre o 
desenvolvimento de um ICALL &  &  &x &x &x &x &x &x &x &x    \\\hline
Desenvolvimento de um 
ICALL                       &  &  &  &x &x &x &x &x &x &x    \\\hline
Elabora��o do Produto de 
PFC1                       &x &x &x &x &  &  &  &  &  &      \\\hline

Elabora��o do Produto de 
PFC2                       &  &  &  &  &  &  &x &x &x &x     \\\hline

\end{tabular}
\end{table}
\end{center}
\newpage

\bibliography{refs}
\bibliographystyle{apalike_prs}
\end{document}